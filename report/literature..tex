%%%---MY REPORT:

For example, most of the current methods project the high-dimensional data into two or three latent components, and the distances in the latent space are interpreted as biological cell-to-cell variability. This assumption might cause misleading results as the dimensionality reduction methods could be sensitive to noise in gene expression data. 


%%%%%%

One of the first tools we came across during out research review was \textit{NetLand}, a software to model and visualise Waddington's epigenetic landscape \cite{netland}. It was designed to quantitatively model, simulate and visualise gene regulatory networks (GRNs) and their corresponding quasi-potential landscapes \cite{netland}; one can import models of GRN files including SBML format and manually adjust the networks structure. For the visualisation one can manually select genes or choose to apply dimensionality reduction. It runs on a Java virtual machine and comes with a nice user interface. From that, we were morivated to gear our literature research more towards the methodology behind it. 

Some state-of-the-art models to generate Waddington's epigenetic landscape that we found during our literature review and helped us to familiarise with the research in this field will be outlined here. A lot of the models use pseudotime estimation to help understand the transition of gene expression profiles. \textit{Pseudotime} is used to measure the progress through a biological process along which cells are arranged based on their expression profiles \cite{hopland}. By using estimated pseudotime of single-cell data, we can identify important regulators by comparing  the expression profiles around the branching time points \cite{hopland}. Due to high variabiliy in the gene expression between cells, recovery of pseudotime is challenging. 

One interesting model that tries to do that and achieves currently the most accurate pseudotime prediction compared to other methods, is \textit{HopLand} which we studied in more detail (see [MH]'s report). It uses a continuous Hopfield network, a type of recurrent neural network, to map cells to a Waddington's epigenetic landscape. From the single-cell data, it tries to reveal the regulatory interaction between genes that control the progression through successive cell states \cite{hopland}. It performs well on identifying key genes and regulatory interactions driving the transition of cell states. The pseudotime is estimated by computing the geodesic distance between every two cells in the landscape. \cite{hopland}. A major advantage to this method is also, that we do not need temporal information or prior knowledge of marker genes \cite{hopland} and we can simulate differentiation processes with multiple lineages. \\

Several other models came before HopLand.
One of them is \textit{Wanderlust} \cite{wanderlust} which uses a graph-based trajectory detection algorithm to align cells on a 1D trajectory on ther developmental path. On the one hand, this approach had a couple of constraints including that cells have to be representative of the whole developmental proces and that trajectories are non-branching. On the other hand, graph-based representation of the data can handle challenges like constructing an accurate trajectory when the relationships between markers cannot be assumed to be linear and a standard metric would result in poor mesaurement of the actual distance in development \cite{wanderlust}.\\
Another interesting model we discovered was \textit{Topslam} \cite{topslam}. It estimates the pseudotime by mapping the individual cells to the surface of the landscape. Similar to HopLand it also uses a probabilitsic dimennsionality reduction technique. The locations of the individual cells reflect their degree of maturity during differentiation as the as they move along the topography of the probabilistic landscape \cite{hopland}. 
To redefine distance, Topslam uses the topography of the landsscape which, however, lacks biological meaning.\\
%Furthermore, although some of the existing data-driven methods could reveal the dynamics of a specific process, they are confined to the identification of key regulators without the involvement of the system dynamics driven by molecular interactions, e.g. reactions among transcription factors, genes and epigenetic modifiers.
Another model called \textit{diffusion map} \cite{diffusionmap} uses diffusion distances to simulate cell differentiation. This way, it can order cells along the differentiation path without losing the non-linear structure of the data.\\ 

Other models include \textit{SCUBA} \cite{scuba} which uses temporal information to perform bifurcation analysis of single-cell data to recover the cell lineages \cite{scuba}. Also \textit{Wishbone} \cite{wishbone} which is based upon Wanderlust aligns single cells into bifurcation branches, it identifies the bifurcation points and recovers the pseudotemportal ordering of cells. \textit{Monocle} \cite{monocle} which builds a minimum spanning tree connecting cells to estimate the pseudotime.
%The differentiation path or landscape is visualised by dimensionality reduction methods (DR).
We analysed these models, with a special interest in how they cope with dimensionality reduction (DR) (see [FB]'s report).\\



As much on the visualisation side of Waddington's landscape has been achieved already (see \cite{netland} and last year's group), we wanted to more focus on the methodology side of Waddington's landscape including different ways to simulate the landscape. Having said that, we wanted to provide our user different tools to simulate and analyse the landscape, or based on single-cell data visualise certain genes or gene combinations after applying dimensionality reduction. We believe this to be a good compromise between more elaborated models that already exist as outlined in section \ref{literaturereview} and the visualisation amongst NetLand and the result from last year's group. In the end, our package is written in Julia and does not require a Java virtual machine as NetLand does.




%and helps construct a distance metric that corresponds to developmental chronology 
%
%.
%
%
%- changes in protein expression are gradual during development
%
%
%Other models regarding analysing the epigenetic landscape have been developed, including bifurcation analysis \cite{bifurcation}
%%, that extracts the lineage relationships from single-cell gene expression data and modelis the dynamic changes associated with cell differentiation.
%
%%%%%%%%


%\subsection{Wanderlust}
%uses a graph-based trajectory detection algorithm to align cells on a 1D trajectory on ther developmental path.
%he pseudotime of a cell is determined by its coordinate on the path. 
%However, it makes several assumptions regarding the data: 
%- Cells have to be representative of the whole developmental process.
%- trajectories are non-branching 
%- changes in protein expression are gradual during development
%CON: 
%it fails to report the divergent time points when there are no branching processes
%it relies on the prior knowledge of marker genes
%
%Ordering single cells onto a trajectory is based on continuous tracking of the progressive rise and fall of phenotypic markers during development. This trajectory provides a framework to infer the order and transition between additional key molecular and cellular events.
%
%A fundamental challenge to constructing an accurate trajectory is that the relationships between markers cannot be assumed to be linear. Thus, determining the distance between two individual cells using standard metrics based on marker levels (e.g., Euclidian norm or correlation) results in poor measures of their chronological distance in development, except in the case of very similar cells. Figure 1A demonstrates the nonlinearity that manifests from using only two markers; while cells X and Y are close based on Euclidian distance, they are quite distant in terms of developmental chronology. The complexity of such nonlinear behavior only increases as more instances occur in high dimensions.
%
%A 
%
%
%
%
%Moreover, because the model is based on similarity between cells, rather than relationships between parameters, it can more naturally handle the nonlinearity.
%



%
%
%\subsection{HopLand}
%- motivation: 
%- application : embryonic cell proliferation
%
%uses single-cell data
%uses interpretation of dynamics in single-cell data to understand the transition of gene expression profiles.





%APPROACH HAS BEEN ANALYSED FURTHER







